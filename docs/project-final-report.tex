\documentclass[letterpaper,twocolumn,10pt]{article}
\usepackage{usenix,epsfig,endnotes,enumerate,graphicx,multicol}
\begin{document}

%don't want date printed
\date{}

%make title bold and 14 pt font (Latex default is non-bold, 16 pt)
\title{\Large \bf DLisp: Automatically distributed computation}

\newcommand{\tuftsauthor}[1]{{\rm #1}\\
  Tufts University}

\author{
  \tuftsauthor{Maxwell Bernstein}
  \and
  \tuftsauthor{Matthew Yaspan}
}

\maketitle

\subsection*{Abstract}
Many of today's programs are written sequentially, and do not take advantage of
the computer's available resources. This is in a large part due to the
difficulty of using any given available threading or parallelization API.

We solve this problem by automatically parallelizing or distributing
computation across cores or even across a datacenter, then analyze the
performance of our distribution algorithms across several modes.

\section{Design}

\subsection{Language}

We began with a Lisp-like language with support for most forms: \verb|fixnum|,
\verb|boolean|, \verb|symbol|, \verb|lambda|, \verb|funcall|, \verb|define|,
\verb|val|, \verb|quote|, \verb|if|, \verb|let|, \verb|let*|, \verb|eval|,
\verb|apply|, built-in functions, and closures. Then we decided that users
would be less comfortable with a Lisp than a programming language with a syntax
that mirrors existing programming languages like SML and OCaml, and changed the
syntax. Mergesort, for example, begin as:

\begin{verbatim}
(define mergesort (xs)
  (if (or (null? xs) (null? (cdr xs)))
    xs
    (let* ((size    (length xs))
           (half    (/ size 2))
           (fsthalf (take half xs))
           (sndhalf (drop half xs)))
      (merge (mergesort fsthalf)
             (mergesort sndhalf)))))
\end{verbatim}

\newpage

and, after the syntax transformation, ended as:

\begin{verbatim}
fun mergesort(xs) =
  if null?(xs) or null?(cdr(xs))
  then xs
  else let* val size = length(xs),
            val half = size/2,
            val fsthalf = take(half, xs),
            val sndhalf = drop(half, xs)
       in merge(mergesort(fsthalf), mergesort(sndhalf))
       end;;
\end{verbatim}

Both programs map to the same abstract syntax tree, which means that we even
allow mixing and matching of styles in the same program, as in:

\begin{verbatim}
fun mergesort(xs) =
  if [or [null? xs] [null? [cdr xs]]]
  then xs
  else let* val size = length(xs),
            val half = size/2,
            val fsthalf = take(half, xs),
            val sndhalf = drop(half, xs)
       in [merge mergesort(fsthalf) mergesort(sndhalf)]
       end;;
\end{verbatim}

One language property that \textit{did not} change in the transition was
mutation; DLisp forces variable immutability. In forms that introduce new
environments, such as \verb|let| and \verb|lambda|, shadowing is allowed -- but
never mutation. This, as it turns out, is key when attempting to parallelize
computation.

For this reason, \verb|map|, \verb|pmap| (parallel map), and \verb|dmap|
(distributed map) are all built-in special forms.

\newpage

\subsection{Network}

We use several terms across the span of this writeup:

\begin{itemize}
    \item \textit{Master} --- Main controller computer from which the program
        is run and distributed. Communicates with several Machines and Workers.
    \item \textit{Machine} --- Logical or physical computer, which contains
        many Agents and Workers.
    \item \textit{WorkPacket} --- A serializable tuple of the form \verb|{Exp|
        $x$ \verb|Env}| that is sent to Workers.
    \item \textit{Worker} --- Process whose sole purpose is to receive
        WorkPackets, evaluate them, and send the results back to the Master.
    \item \textit{SlowWorker} --- A Worker that has been artificially slowed
        down by a constant factor.
    \item \textit{Agent} --- Process whose sole purpose is to manage a work
        queue.
    \item \textit{StealingAgent} --- Agent that occasionally steals from other
        Agents when its worker is moving quickly.
    \item \textit{RoundRobinMode} --- Distribution mode that uses a circular
        queue to hand out work to Workers in order.
    \item \textit{ByMachineMode} --- Distribution mode that uses per-Machine
        statistics to determine which Worker should receive a given WorkPacket.
        Currently, this has two sub-modes:
        \begin{itemize}
            \item \textit{Timed} --- Hand out work to whoever can respond the
                fastest to a HealthCheck.
            \item \textit{Memory} --- Hand out work to whoever has the most
                computational power (currently measured by memory pressure)
                currently available.
        \end{itemize}
\end{itemize}

We use normal (non-stealing) Agents to begin with, then proceed to demonstrate
the utility and speed gains by using StealingAgents. Additionally, we introduce
some SlowWorkers into the Worker pool to demonstrate that work stealing is an
effective means of combating heterogeneous computational power.

Additionally, we demonstrate the results of different distribution modes
(enumerated above) and their affects on end-to-end computation speed.

\subsection{Startup Procedure}

A Master node is started, and runs on node \verb|M|. Independently, anywhere
from 0 to N>0 Machines are started up on the same network, with knowledge of
the Master. In this case, we use Erlang nodenames, such as the atom
\verb|master@some.ip.address.here|, to identify the Master.

\end{document}
